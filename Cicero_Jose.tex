\documentclass[10pt]{article}
\usepackage[utf8]{inputenc}
\usepackage{graphicx}

\title{CC0602 - Computação Gráfica}
\author{Cícero José}
\date{Abril 2019}

\begin{document}

\maketitle

\section{Introdução}
\quad De acordo com o Professor Fernando Siqueira \cite{ProfessorFernandodeSiqueira}. A computação gráfica é a área da Ciência da Computação que trata da produção e representação de informações através de imagens, animações e vídeos com o uso de recursos computacionais.

\quad Segundo a ISO ("International Standards Organization") a Computação Gráfica pode ser definida como o conjunto de métodos e técnicas utilizados para converter dados para um dispositivo gráfico, via computador.

\quad A Computação Gráfica permite a representação visual da informação em situações em que uma imagem é capaz de representar algo (conceitos, instruções etc.) sem a utilização de palavras pois consegue transmitir ideias de forma mais precisa, rica e eficiente do que outras formas de representação da informação.

\quad A computação gráfica pode ser entendida como o conjunto de algoritmos, técnicas e metodologias para o tratamento e a representação gráfica de informações através da criação, armazenamento e manipulação de desenhos em formato digital, utilizando-se computadores e periféricos gráficos.

\quad É portanto uma sub-área da ciência da computação que trata de representar informações e dados através de imagens, desenhos e vídeos gerados e manipulados através do computador.

\section{Relevância}
\quad Segundo os autores Isabel Harb Manssour e Marcelo Cohen \cite{Isabel}, a Computação Gráfica é bastante necessária no estudo da Computação, pois essa área possui várias aplicações. Atualmente a Computação Gráfica está presente em quase todas as áreas do conhecimento humano, da engenharia que utiliza as tradicionais ferramentas CAD (Computer-Aided Design), até a
medicina que trabalha com modernas técnicas de visualização para auxiliar o diagnóstico por
imagens. Nesta área, também têm sido desenvolvidos sistemas de simulação para auxiliar no
treinamento de cirurgias endoscópicas. Outros tipos de simuladores são usados para treinamento de pilotos e para auxiliar na tomada de decisões na área do direito (por exemplo, para
reconstituir a cena de um crime).
\quad Na figura a seguir observa-se a capacidade de construir ou reconstruir imagens e cenários utilizando a Computação Gráfica.\newline 

\begin{figure}[htb]
    \centering
    \includegraphics[width= 6cm, height=5cm]{Glassy.jpg}
    \includegraphics[width= 6cm, height=5cm]{Alexexterior3.jpg}
    \caption{Exemplos de figuras geradas completamente por Computação Gráfica\cite{glassy}.}
    \label{fig:my_label}
\end{figure}

\section{Relação com outras disciplinas}

\begin{table}[htb]
\begin{tabular}{|l|l|}
\hline
\begin{tabular}[c]{@{}l@{}}CC0301\\ Algoritmos e Estruturas de Dados II\end{tabular} & \begin{tabular}[c]{@{}l@{}}Computação Gráfica é uma disciplina com ênfase\\  na implementação de algoritmos relativamente complexos e o \\ aluno deve possuir bons conhecimentos de estruturas \\ de dados e boa experiência de programação. Para que, assim,\\  ele possua um bom desempenho na disciplina.\cite{lapix}\end{tabular} \\ \hline
\begin{tabular}[c]{@{}l@{}}CC0406 \\ Cálculo Numérico\end{tabular}                   & \begin{tabular}[c]{@{}l@{}}Na disciplina de Computação Gráfica é requerida \\ uma boa base de Cálculo Numérico, \\ pois é necessário lidar com os recursos que \\ essa disciplina proporciona para a resolução de problemas.\end{tabular}                                                                                   \\ \hline
\end{tabular}
\end{table}
\bibliographystyle{ieeetr}
\bibliography{Cicero_Jose}

\end{document}
