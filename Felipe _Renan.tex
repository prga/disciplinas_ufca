\documentclass[10pt]{article}
\usepackage[utf8]{inputenc}
\usepackage{graphicx}
\title{Projeto 2}
\author{Felipe Renan }
\date{Abril 2019}

\begin{document}

\maketitle{Aprendizado de máquina.}
\section{Introdução.}\

\begin{flushleft}
\setlength{\baselineskip}{1.5\baselineskip}

Este breve resumo aborda um importante e crescente campo do conhecimento dentro da ciêcia da computação, o aprendizado de máquina (ou machine learning), que consiste, basicamente, em fazer com que a máquina comece a aprender por conta própria, com a ajuda de alguns comandos específicos previamente feitos por um programador para que a máquina aprenda a "pensar" por conta própria.
\end{flushleft}
\section{Relevância.}


\begin{flushleft}
\setlength{\baselineskip}{1.3\baselineskip}
O aprendizado de máquina tem aparecido cada vez mais depois do exponecial avanço da tecnologia no mundo, processadores muito mais potentes e capazes de realizar diversas tarefas ao mesmo tempo vem surgindo a cada curto espaço de tempo. De forma mais técnica: ``Machine Learning da maneira mais básica é a prática de usar algoritmos para coletar dados, aprender com eles, e então fazer uma determinação ou predição sobre alguma coisa no mundo.''~\cite{ref1}, sendo assim,  tal aprendizado auxilia os seres humanos em tarefas que demandariam muito esforço e repetição, o que para um computador, se torna uma tarefa menos árdua de ser realizada, como processar uma grande amostra de dados, ou verificar certos padrões em mensagens e caracterizá-las como spams. 
\end{flushleft}
\begin{figure}[h]
    \centering
    \includegraphics[scale = 0.3]{spam_fit.jpg}
    \caption{Aprendizado de maquina usado pra detectar spams}
    \label{fig:my_label}
\end{figure}
\section{Relação com outras disciplinas.}
\begin{table}[htb]
\centering
\label{tabela}
\begin{tabular}{|c|c|}
\hline
\textbf{Disciplina necessária}  & \textbf{Importância}                                                                                                                                                                                                                                     \\ \hline
Inteligência artificial(CC0671) & \begin{tabular}[c]{@{}c@{}}Mostra os primeiros passos para o desenvolvimento \\ de técnicas para a criação de algoritmos que possam\\ viabilizar e realizar o aprendizado da máquina por si\\ própria, dando uma boa base para tal processo.\end{tabular} \\ \hline
\end{tabular}
\end{table}

\begin{flushleft}
\setlength{\baselineskip}{1.3\baselineskip}
Como pôde ser visto na tabela \ref{tabela}, a disciplina necessária parar iniciar o estudo da Aprendizagem de máquina é a disciplina de inteliência artificial, pois é nela que se aprende os primeiros conceitos sobre como fazer a máquina pensar por sí, desenvolvendo programas e aprendendo mais sobre a lógica mais profunda de programação. 


\end{flushleft}
\bibliographystyle{ieeetr}
\bibliography{referencias.bib}
\end{document}
