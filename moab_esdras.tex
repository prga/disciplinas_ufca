\documentclass[10pt]{article}
\usepackage[utf8]{inputenc}

\title{CC0652 - Programação para WEB}
\author{Moab Esdras}
\date{Abril 2019}

\begin{document}

\maketitle

\section{Introdução}
\quad As áreas no ramo de TI são vastas,e a programação voltada à web baseia-se na interação entre um servidor e seu cliente visando a melhor distribuição e uso do produto, a programação web é importante na criação de sites de lojas físicas e criação de lojas virtuais.Segundo o site Host Gator\cite{ref2} O sucesso de um projeto web depende muito da participação de todos os envolvidos. Por isso, é fundamental que o desenvolvedor deixe isso bem claro ao cliente e explique todos os passos que serão realizados. Porém, na correria do dia a dia é comum que o planejamento seja deixado de lado, o que pode comprometer o andamento e a entrega do projeto. 

\quad De acordo com o site Portal Educação\cite{ref1}O mercado web é povoado abrangentemente pela linguagem Java. Criar um projeto em Java é bem mais amplo e independente de vários softwares como servlet container, banco de dados e até da própria fabricante da sua Virtual Machine. Mas trabalhar com a linguagem Java na web não é tão trivial. O desenvolvedor web precisa conhecer todas as características de APIs, de servlets, de JSP e de HTTP.

\quad Fazendo uso da linguagem java, o desenvolvimento dessas aplicações se dá de forma distribuída e controlada, de modo que uma informação é processada em um servidor, e executadas no dispositivo do usuário. Existem componentes que desenvolvedores web utilizam visando o desenvolvimento dessas ferramentas, tais componentes ficam responsáveis pelo armazenamento temporário dos dados que posteriormente serão transitados em cada resposta em uma página na web.Essas aplicações fazem uso principalmente de linguagens como, Html,JavaScript, e CSS.

\section{Relevância}
\quad A web faz parte da vida de todo mundo atualmente, a criação de sites na web diariamente, é imensa, em um endereço o usuário decide o que quer ver, acessar, e buscar de forma simples e direta, tendo como base suas permissões e funções dentro daquele site. Tudo isso dentro de uma interface gráfica funcional que auxilia seus comandos aos já fornecidos ao site por um programador web.Além disso, o uso dos navegadores pela a maioria dos usuários melhora a adaptação dos mesmos, tudo isso aliado a centralidade das manutenções e atualizações.Diferentemente de aplicações tradicionais, que precisam ser instaladas para todos os usuários, aumentando assim a complexidade de conexões entre aplicações de um determinado SI e outro,desse modo, uma aplicação centrada no dispositivo em que será instalada, poderá resultar em conflitos com outros softwares ou até mesmo com o hardware,na contra-mão dessa ideia se encontra os websites.Segundo Caio Augusto Amorim\cite{ref3} estar nos conformes da criação de Sites e Aplicações Web deixou de ser uma opção dispendiosa há algum tempo. Trata-se de uma necessidade para sobrevivência. Termos como navegabilidade, usabilidade, tableless, css, Javascript não-obtrusivo etc, devem fazer parte do vocabulário de qualquer um que arrisque autodenominar-se desenvolvedor.

\section{Relação com outras disciplinas}
\quad A tabela a seguir mostra um breve resumo da relação entre a disciplina de programação orientada a web e a programação orientada a objeto.
\begin{table}[htb]
\label{tabela}
\begin{tabular}{|l|l|}
\hline
CC00402 & \begin{tabular}[c]{@{}l@{}}A linguagem Java é uma das principais quando o assunto é \\ Programação Orientada a Objeto,\\ dessa forma há uma ligação interessante entre Java e Javascript,\\  que é um dos pilares da programação web,\\ essa relação interliga os conteúdos dados nas duas disciplinas,\\ visto que a programação orientada a objeto é\\  pré-requisito para a disciplina de programação web.\end{tabular} \\ \hline
\end{tabular}
\end{table}
\bibliographystyle{ieeetr}
\bibliography{Moab}
\end{document}