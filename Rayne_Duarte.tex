\documentclass[10pt]{article}

\usepackage{graphicx}

\usepackage[utf8] {inputenc}


\title{Inteligência Artificial}

\author{Raynê Duarte}

\date{Abril 2019}


\begin{document}

\begin{figure}

\centering

\includegraphics[width = 12cm, height=12cm]{Sophie.jpg}

\caption{Sophie, a primeira robô do mundo com inteligência artificial a ganhar cidadania oficial em um país}

\label{fig:Sophie.jpg}

\end{figure}

\maketitle

\newpage

\section*{Introdução}
Inteligência Artificial ou A.I do inglês Artificial Inteligence, é o nome dado ao conjunto de técnicas e estudos que têm como objetivo fazer com que as máquinas, mais precisamente softwares tenham a capacidade de tomar decisões mediante instruções pré-definidas, de modo que essas decisões não sejam pré-definidas, mas sim baseadas em instruções tanto internas ao software, quanto externas, ou seja, situações ou condições que incentivem a máquina a decidir o que fazer, semelhante ao raciocínio lógico de um ser humano.


\section*{Uma breve história do conceito de A.I}

Herbert Simon, Jonh McCarthy e Allen Newell foram os pioneiros no desenvolvimento do conceito de A.I, tendo como objetivo inicial a criação de uma máquina que conseguisse simular a vida humana. O termo A.I surgiu em 1950, mas passou a ser reconhecido como uma ciência somente em 1956, ano em que foi fundado o primeiro campo de pesquisa de inteligência artificial em uma conferência no campus do Dartmouth College. O interesse pela AI cresceu a tal ponto que deu origem a filmes, documentários e até mesmo livros científicos como "Inteligência Artificial" de \cite{r1}Stuart J. Russel e \cite{r2}Peter Norvig, sendo uma das referências bibliográficas mais famosas acerca do assunto. 


\section*{Relevância: A Importância da AI no curso de Ciência da Computação}

O estudo da A.I no curso de Ciências da Computação faz se necessário para preparar o aluno para o desenvolvimento de softwares, teorias e aplicações no mercado de trabalho que fazem o uso desta tecnologia. A existência da Inteligência Artificial como disciplina na grade curricular é de fundamental importância para que o aluno como programador saiba como fazer um melhor uso das linguagens de programação a fim de criar algoritmos capazes de solucionar problemas por meio da análise de dados. Programar não é apenas conhecer os comandos e funções de uma linguagem, mas sim saber como fazer um uso eficiente destes em aplicações  críticas e que envolvem a tomada de decisões, e ao meu ver, o estudo da A.I incentiva o aluno a ter ideias que auxiliem no uso eficiente destes comandos e funções.

\newpage
\section*{Relações com outras disciplinas}
A tabela a seguir \ref{tabela1}mostra a relação de 2 disciplinas com o estudo
da AI no curso, bem como a importância de cada uma no campo da Inteligência
Artificial.


\begin{table}[!htb]
\label{tabela1}
\begin{tabular}{|c|c|}
\hline
CC0671- Aprendizado de máquina                                                             & \begin{tabular}[c]{@{}c@{}}É um subcampo da AI que está relacionado com\\ a mesma principalmente devido ao uso de suas\\ técnicas de aprendizado supervisionado e não\\ supervisionado, as quais são bastante úteis em\\ áreas como previsão, análise e mineração de dados.\end{tabular}                                                                                                                                                                                                                                                                                                                                                                                                                             \\ \hline
\begin{tabular}[c]{@{}c@{}}CC0872- Processamento de\\ Linguagem Natural (PLN)\end{tabular} & \begin{tabular}[c]{@{}c@{}}Uma subárea da AI que tem como objetivo entender as\\ limitações de uma máquina ou software em compreender a\\ linguagem humana.\\ Basicamente a PLN busca fazer com que softwares sejam\\ capazes de interpretar textos, extrair informações destes, desenvolver\\ conceitos e opiniões, e até mesmo analisar sentimentos de acordo\\ com o conteúdo desses textos.\\ Tendo em vista os critérios estudados pela PLN, não há dúvidas\\ de que entre todos os campos da ciência que investigam a AI,\\ a PLN é a que mais pode contribuir para a criação de máquinas\\ com padrões de comportamento, inteligência e semelhanças\\ cada vez mais próximas a de um ser humano.\end{tabular} \\ \hline
\end{tabular}
\end{table}

\bibliographystyle{ieeetr}
\bibliography{references}
\end{document}

