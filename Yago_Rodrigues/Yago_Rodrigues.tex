\documentclass{article}
\usepackage[utf8]{inputenc}
\usepackage{graphicx}
\title{Algoritmos e Estrutura de Dados I}
\author{Yago Rodrigues }
\date{April 2019}

\begin{document}

\maketitle
\newpage
\section{Introdução}
Algoritmos e estrutura de dados I é uma disciplina fundamental dentro do curso de ciência da computação, é um modo particular de armazenamento e organização de dados em um computador, de modo que possam ser usados com eficiência.

Ela nos apresenta diversas estruturas fundamentais, como estruturas lineares e estruturas não lineares (árvores), os   algoritmos   básicos   para   a   sua   manipulação,   assim   como   as   suas aplicações.

Durante o desenvolvimento de um software, cada método que será utilizado deve ser analisando, deve-se verificar sua complexidade e seus impactos nodesempenho. Deve-se saber o que será mais importante, a velocidade ou a estabilidade.
Se uma coleção de dados está salva na memória, provavelmente essas informações serão usadas posteriormente, para isso precisaram ser recuperadas e termos que utilizar métodos de busca.
Quando os dos encontra-se já ordenados o processo de busca se torna muito eficiente, então ordenar uma coleção de elementostorna-se muito importante. Embora se tenha muitos métodos que realiza essa tarefa, existem uma diferença muito grande entre eles, principalmente em relação ao desempenho. Além disso, essa coleção dados tem que está organizada.
A organização de uma coleção de dados pode ser feita através de vetores estáticos ou listas dinâmicas.

A figura \ref{imagem} mostra um exemplo de como é utulizada a estrutura de daos para ordenção de informações.
\begin{figure}[!htb]
     \centering
     \includegraphics[scale=0.8]{lifo_stack.png}
     \caption{Estrutura de dados: Pilha}
     \label{imagem}
\end{figure}
\newpage
\section{Relevância}
De acordo com \cite{dsi} A  resolução  de  um  problema  através  de  um  algoritmo  e  consequente  programa computacional  refere-se  ao  processo  de  identificar  e  analisar  um  problema  do mundo  real  e  desenvolver  a  sua  solução  de  modo  eficiente.  Este  processo  é constituído pelas seguintes fases: (1) identificação e compreensão do problema (e objetivos),  (2)  conceitualização  da  solução,  (3)  definição  do  algoritmo  para  a resolução  do  problema,  e  (4)  implementação  (codificação)  da  solução  através  de um programa computacional. A tarefa de escrever um algoritmo pode ser simplificada através da decomposição e   análise   de   subproblemas.   O   processo   de   estruturação   na   resolução   de problemas reflete-se num programa modular constituído por diferentes partes que definem a solução do problema.
\newpage
\section{Relação com outras disciplinas}
A tabela \ref{tabela} com dados retirados de \cite{UFCA} mostra a relação que as disciplinas tem com Algoritmo e estrutura de dados I, direta e indiretamente.
\begin{table}[htb]
\label{tabela}
\begin{tabular}{|l|l|}
\hline
\begin{tabular}[c]{@{}l@{}}CC0102\\ Introdução à Programação\end{tabular}                       & \begin{tabular}[c]{@{}l@{}}É PRÉ REQUISITO PARA ALGORITMO\\  E ESTRUTURA DE DADOS I\end{tabular}     \\ \hline
\begin{tabular}[c]{@{}l@{}}CC0301\\ Algoritmos e Estruturas de \\ Dados II\end{tabular}         & \begin{tabular}[c]{@{}l@{}}TEM ALGORITMO E ESTRUTURA \\ \\ DE DADOS I COMO PRE REQUISTO\end{tabular} \\ \hline
\begin{tabular}[c]{@{}l@{}}CC0401\\ Algoritmos em Grafos\end{tabular}                           & \begin{tabular}[c]{@{}l@{}}TEM ALGORITMO E ESTRUTURA \\ \\ DE DADOS I COMO PRE REQUISTO\end{tabular} \\ \hline
\begin{tabular}[c]{@{}l@{}}CC0501\\ Projeto e Análise de A\\ lgoritmos\end{tabular}             & \begin{tabular}[c]{@{}l@{}}TEM ALGORITMO E ESTRUTURA \\ \\ DE DADOS I COMO PRE REQUISTO\end{tabular} \\ \hline
\begin{tabular}[c]{@{}l@{}}CC0601\\ Autômatos, Computabilidade \\ e Complexidade\end{tabular}   & \begin{tabular}[c]{@{}l@{}}TEM ALGORITMO E ESTRUTURA \\ \\ DE DADOS I COMO PRE REQUISTO\end{tabular} \\ \hline
\begin{tabular}[c]{@{}l@{}}CC0701\\ Compiladores\end{tabular}                                   & \begin{tabular}[c]{@{}l@{}}TEM ALGORITMO E ESTRUTURA \\ \\ DE DADOS I COMO PRE REQUISTO\end{tabular} \\ \hline
\begin{tabular}[c]{@{}l@{}}CC0302\\ Laboratório de Programação\end{tabular}                     & \begin{tabular}[c]{@{}l@{}}TEM ALGORITMO E ESTRUTURA \\ \\ DE DADOS I COMO PRE REQUISTO\end{tabular} \\ \hline
\begin{tabular}[c]{@{}l@{}}CC0404\\ Programação Concorrente\end{tabular}                        & \begin{tabular}[c]{@{}l@{}}TEM ALGORITMO E ESTRUTURA \\ \\ DE DADOS I COMO PRE REQUISTO\end{tabular} \\ \hline
\begin{tabular}[c]{@{}l@{}}CC0504\\ Sistemas O\\ peracionais\end{tabular}                       & \begin{tabular}[c]{@{}l@{}}TEM ALGORITMO E ESTRUTURA \\ \\ DE DADOS I COMO PRE REQUISTO\end{tabular} \\ \hline
\begin{tabular}[c]{@{}l@{}}CC0602\\ Computação Gráfica\end{tabular}                             & \begin{tabular}[c]{@{}l@{}}TEM ALGORITMO E ESTRUTURA \\ \\ DE DADOS I COMO PRE REQUISTO\end{tabular} \\ \hline
\begin{tabular}[c]{@{}l@{}}CC0402\\ Programação Orientada a \\ Objetos\end{tabular}             & \begin{tabular}[c]{@{}l@{}}TEM ALGORITMO E ESTRUTURA \\ \\ DE DADOS COMO PRE REQUISTO\end{tabular} \\ \hline
\begin{tabular}[c]{@{}l@{}}CC0503\\ Banco de Dados\end{tabular}                                 & \begin{tabular}[c]{@{}l@{}}TEM ALGORITMO E ESTRUTURA \\ \\ DE DADOS I COMO PRE REQUISTO\end{tabular} \\ \hline
\begin{tabular}[c]{@{}l@{}}CC0405\\ Fundamentos de \\ Linguagens \\ de Programação\end{tabular} & \begin{tabular}[c]{@{}l@{}}TEM ALGORITMO E ESTRUTURA \\ \\ DE DADOS I COMO PRE REQUISTO\end{tabular} \\ \hline
\end{tabular}
\end{table}
\newpage

\bibliographystyle{ieeetr}
\bibliography{Referencias.bib}

\end{document}
