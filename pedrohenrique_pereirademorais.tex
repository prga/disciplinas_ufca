\documentclass[10pt, twocolums]{article}
\usepackage[utf8]{inputenc}
\usepackage{graphicx}

\title{CC0202-Lógica aplicada à computação}
\author{Pedro Henrique Pereira De Morais }
\date{Abril 2019}

\begin{document}

\maketitle

\section*{Introdução}

\paragraph{} Apresentar conceitos e teoremas de lógica de primeira ordem clássica, seus e usos; diferentes
métodos de prova para lógica proposicional, de forma que o aluno possa trabalhar com cálculo
proposicional; conceitos e elementos da lógica de primeira ordem; conceitos e elementos básicos
da linguagem Prolog e apresentar o Paradigma de programação em Lógica.
\section*{Relevância}
\paragraph{}É importante para que o aluno após cursar a disciplina, seja capaz de:
\begin{itemize}
    \item Reconhecer e trabalhar com os símbolos formais que são usados nas lógicas proposicional e de primeira ordem;
    \item Avaliar o valor-verdade de uma expressão na lógica proposicional;
    \item Avaliar o valor-verdade de uma fórmula de primeira ordem em alguma interpretação;
    \item Usar a lógica proposicional e a lógica de primeira ordem para representar e avaliar argumentos;
    \item Construir demonstrações formais nas lógicas proposicionais e de primeira ordem e usá-las para determinar a validade de um argumento (ou a solução de um problema).
\end{itemize}
\pagebreak
\section*{Relação com outras disciplinas}
\begin{table} [h]
\begin{tabular}{|l|l|} 
\hline
Disciplina               & Importancia da lógica aplicada a computação                                                                                                              \\ \hline
Introdução à programação & \begin{tabular}[c]{@{}l@{}}Ajuda ao aluno a estabelecer pensamentos lógicos\\  para melhor criação de programas\end{tabular}                             \\ \hline
Circuitos Digitais       & \begin{tabular}[c]{@{}l@{}}Se relaciona a circuitos digitais pela necessidade de\\ pensamentos lógicos para a criação de circuitos digitais\end{tabular} \\ \hline
Inteligência artificial  & \begin{tabular}[c]{@{}l@{}}Melhora na criação de IAs ajudando o aluno a fazer IAs\\ mais eficientes atraves da lógica\end{tabular} \\ \hline
\end{tabular}
\end{table}
\begin{figure}[b]
\centering
\includegraphics [scale= 0.35] {foto.jpg}
\label{Rotulo}
\end{figure}
\section*{Referências}
\kill \cite{1}
\kill \cite{2}
\kill \cite{3}
\bibliographystyle{ieeetr}
\bibliography{refss.bib}
\end{document}
