\documentclass[10pt]{article}
\usepackage[brazil]{babel}
\usepackage{graphicx}
\graphicspath{{figuras/}}
\usepackage[utf8]{inputenc}
\usepackage{hyperref}
\usepackage{cite}
\hypersetup{
    colorlinks=true,
    linkcolor=black,
    filecolor=black,      
    urlcolor=black,
    pdftitle={Engenharia de Software},
    bookmarks=true,
    pdfpagemode=FullScreen,
    }
\urlstyle{same}
\title{CC0026 - Engenharia de Software}
\author{Pedro Henrique Lopes dos Santos}
\date{Onze de Abril de Dois mil e Dezenove}

\begin{document}

\maketitle

\section{Introdução}
  \begin{figure}[!htb]
    \centering
    \includegraphics[scale=0.4]{figura1.png}
    \caption{Imagem Ilustrativa onde está escrito Engenharia de Software em Inglês}
    \label{fig1}
  \end{figure}
    A Engenharia de Software é uma disciplina obrigatória do curso de Ciência da Computação ofertado pela 
UFCA - Campus Juazeiro do Norte que tem como objetivo apresentar as técnicas, métodos e ferramentas utilizadas no desenvolvimento e manutenção de  Softwares. \cite{PPCcc3} Assim como desenvolver também, estudos de casos baseados no que foi ensinado. Geralmente, o componente é direcionado ao desenvolvimento e manutenção de sistemas de softwares eficientes e confiáveis, aplicando técnicas de engenharia para a gestão das tarefas.
    Com cada vez mais empresas automatizando os serviços e criando suas plataformas próprias, a
engenharia de software foi considerada a segunda melhor da área de TI em crescimento e remuneração, segundo o site especializado em carreiras de tecnologia, \href{https://it.careercast.com/jobs-rated/2017-best-jobs-it?page=1}{CareerCast}.
\cite{eng2}
    Basicamente, estes profissionais projetam e guiam o desenvolvimento de programas, aplicativos e
sistemas, de forma que atendam aos requisitos e cumpram as funções determinadas.
    Dentre as principais atribuições destes profissionais, estão: Desenvolver softwares e apps, Gerenciar
projetos ligados aos softwares, Arquitetar o design estrutural dos programas e realizar testes nos sistemas. 
Além destas, estes engenheiros podem ter funções ligadas à administração de bancos de dados, manutenção dos sistemas e até algumas de documentação, relacionadas à gestão de projetos e à composição dos manuais de instruções. \cite{eng1}

\section{Relevância}
    A importância de se estudar a engenharia de software está na possibilidade de ampliar opções para o
mercado de trabalho quando o estudante estiver formado. Como foi dito no tópico anterior, há muita oferta e pouca procura, tendo como prova da quase inexistente procura a informação de que na Microrregião do Cariri Cearense, têm apenas dois cursos de graduação focados na área. \cite{PPCcc1}
    A Engenharia de Software é uma área da computação voltada à especificação, desenvolvimento e manutenção
de sistemas de software, em relação a todos os aspectos da produção de software. É ela que irá dar os caminhos necessários para o engenheiro de sistemas trilhar, seguindo especificações vindas do cliente, e também as normas e bons costumes que a mesma emprega como modelos de sistemas. Os fundamentos da Engenharia de Software envolvem o uso de modelos precisos que permitem especificar, projetar, implementar e manter o sistema, avaliando e dando garantia de sua qualidade. Em resumo, é uma matéria que serve de base para o começo da criação de um sistema, sendo este qualquer que seja, para qualquer tipo de cliente. Engenharia de Software é essencial e praticamente obrigatória na construção de um sistema de software, pois ela guia o engenheiro desde as primeiras entrevistas feitas com o cliente até a entrega do sistema e a manutenção do mesmo. \cite{eng3}

\section{Relação com as outras disciplinas}
O componente curricular "Engenharia de Software" do curso de Ciência da Computação na UFCA é a última cadeira de uma árvore de disciplinas vistas nos semestres anteriores, tendo como pré-requisito Programação Orientada a Objetos, mas que pelo componente ter pré-requisitos, os alunos de engenharia de software obrigatoriamente têm que ter sido aprovados em Introdução à Programação, Algoritmos e Estruturas de dados I, Algoritmos e Esturutas de Dados II, Laboratório de Programação e enfim Programação Orientada a Objetos. \cite{PPCcc2} O que foi dito é melhor expressado pela Tabela \ref{tabela1}

\begin{table}[!htb]
\label{tabela1}
\begin{tabular}{|l|l|}
\hline
\begin{tabular}[c]{@{}l@{}}CC0402 - Programação Orientada a\\ Objetos\end{tabular}     & \begin{tabular}[c]{@{}l@{}}Pré-Requisito para\\ Engenharia de Software\end{tabular}              \\ \hline
CC0302 - Laboratório de Programação                                                    & \begin{tabular}[c]{@{}l@{}}Pré-Requisito para\\ Programação Orientada a Objetos\end{tabular}     \\ \hline
\begin{tabular}[c]{@{}l@{}}CC0301 - Algoritmos e Estruturas de\\ Dados II\end{tabular} & \begin{tabular}[c]{@{}l@{}}Pré-Requisito para\\ Programação Orientada a Objetos\end{tabular}     \\ \hline
\begin{tabular}[c]{@{}l@{}}CC0201 - Algoritmos e Estruturas de\\ Dados I\end{tabular}  & \begin{tabular}[c]{@{}l@{}}Pré-Requisito para\\ Algoritmos e Estruturas de Dados II\end{tabular} \\ \hline
CC0102 - Introdução à Programação                                                      & \begin{tabular}[c]{@{}l@{}}Pré-Requisito para\\ Algoritmos e Estruturas de Dados I\end{tabular}  \\ \hline

\end{tabular}
\end{table}

\bibliographystyle{ieeetr}
\bibliography{references.bib}

\end{document}