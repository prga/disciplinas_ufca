\documentclass[10pt]{article}
\usepackage[utf8]{inputenc}
\usepackage{graphicx}

\title{CC0502 - Redes de computadores}
\author{Lucas Lemos Monteiro}
\date{April 2019}

\begin{document}

\maketitle

\section{Introdução}
A disciplina de Redes de Computadores faz parte da grade curricular obrigatória de qualquer curso superior da área de informática: ciência da computação, engenharia da computação, tecnologia em análise e desenvolvimento de sistemas, entre outros. O que pode variar de um curso para outro é o modo como ela é abordada. A abordagem principal é basicamente fundamentos teóricos da comunicação de dados e procura situá-los no contexto de uma rede de computadores, estabelecendo o conceito de comunicação e defindo quais os elementos necessários para que esta comunicação ocorra.
\begin{figure}[!htb]
\center{\includegraphics[width=0.8\textwidth]{imagem.jpg}}
\end{figure}
\section{Relevância}
Hoje a matéria de Redes de Computadores, é de extrema importância para os profissionais que desejam atuar nas empresas na área de Tecnologia da Informação (TI). O profissional será responsável por implantar e implementar as melhores soluções para gerenciar o ambiente nas empresas. As redes não estão presentes somente nas grandes organizações, mas também em nossas residências onde é comum o compartilhamento de informações, compartilhar Internet com outros dispositivos.\cite{redes}
\section{Relação com outras diciplinas}
A tabela a seguir faz a relação entre a diciplina de redes de computadores e outras diciplinas do curso de ciência da computação informadas no PPC (Projeto pedagógico do curso)\cite{ppc}.
\begin{table}[!htb]
\label{tabela}
\begin{tabular}{|l|l|}
\hline
\multicolumn{1}{|c|}{\textbf{Diciplinas}} & \multicolumn{1}{c|}{\textbf{Relações com redes de computadores.}}                                                                                               \\ \hline
CC0862 - Segurança de redes               & \begin{tabular}[c]{@{}l@{}}Rede de computadores é a base para o estudo de\\ segurança de redes onde o foco está na troca \\ segura de informações.\end{tabular} \\ \hline
CC0861 - Redes sem fio                    & \begin{tabular}[c]{@{}l@{}}Redes sem fio é uma extensão da rede de\\ computadores onde o foco é a comunicação\\ sem fio.\end{tabular}                           \\ \hline
CC0762 - Avaliação de Desempenho de Redes & \begin{tabular}[c]{@{}l@{}}A avaliação do desempenho é uma ferramenta\\ para medir uma rede de computadores.\end{tabular}                                       \\ \hline
CC0652 - Programação para Web             & \begin{tabular}[c]{@{}l@{}}Programação para Web envolve a comunicação\\ entre  computadores\end{tabular}                                                        \\ \hline
CC0661 - Laboratório de redes             & \begin{tabular}[c]{@{}l@{}}Em laboratório de redes o conteúdo de redes de\\ computadores é colocado em prática\end{tabular}                                     \\ \hline
\end{tabular}
\end{table}

Nessa sentido a diciplina de redes de computadores tem uma ligação com outras diciplinas do curso pois ela estuda a comunicação entre computadores e isso pode ser usado em várias outras áreas.

\bibliographystyle{ieeetr}
\bibliography{main.bib}


\end{document}
