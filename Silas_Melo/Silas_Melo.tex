\documentclass[10pt]{article}
\usepackage[utf8]{inputenc}
\usepackage{graphicx}

\title{CC0502 - Rede de computadores}
\author{Silas Araujo de Melo}
\date{April 2019}

\begin{document}

\maketitle

\section{Introdução}
Com a globalização as organizações estão geograficamente dispersas, com
escritórios em diversas partes do planeta. Estes escritórios necessitam
compartilhar recursos, trocar dados e informações com diferentes freqüências.
As Redes de Computadores permitem que diversos serviços e recursos possam
ser compartilhados, otimizam as comunicações e permitem maior interação
entre os usuários. O conhecimento dos conceitos de Redes permitem aos
profissionais das áreas de tecnologia da informação os fundamentos para
analisar, desenvolver e utilizar os diversos recursos existentes nestas
tecnologias.
A disciplina de redes de computadores apresenta as características sobre
arquiteturas, topologias, protocolos e serviços fornecendo ao aluno
conhecimentos básicos sobre as redes de computadores e seu funcionamento 
\begin{figure}[!htb] 
\center{\includegraphics[width=0.8\textwidth]{123.png}}
\end{figure}
\section{Relevância}
A importância das redes de computadores vem crescendo progressivamente com o passar dos anos. Hoje em dia as redes de computadores são encontradas em todos os lugares desde uma residência familiar até grandes empresas multinacionais isso acontece devido sua grande versatilidade, pois ela pode ser utilizada por uma variedade de aplicações que é capaz de atender as necessidades de cada usuário. O grande objetivo das redes, e o que a torna tão importante é ela fazer com que múltiplos usuários em distâncias indeterminadas compartilhem um determinado recurso. O próprio Andrew S. Tanenbaum, pesquisador muito conceituado, bacharel em ciências da computação pelo M.I.T e P.h.D. pela University of Califórnia, define o objetivo das redes de computadores como “tornar todos os programas, equipamentos e especialmente dados ao alcance de todas as pessoas na rede, independente da localização física do recurso e do usuário”.

Então vemos que recursos tão proveitosos, se usados de maneira inteligente, tem uma grande importância para o avanço da sociedade, pois sua capacidade de aproximação de pessoas, cidades, culturas, etc., acaba tornando o mundo ainda mais globalizado em todos os aspectos.
\section{Relacão com outras diciplinas}
A tabela a seguir irá sitar algumas diciplinas relacionadas a rede de computadores e suas ligações.Referentes ao PPC (projeto pedagógico do curso) \cite{ppc}
Alctel \cite{Alctel}
Impotância\cite{Import}
\begin{table}[htb]
\label{tabela}
\begin{tabular}{|l|l|}
\hline
Diciplinas                                    & Relações com Redes de Computadores                                                                                                                                                                                                                                                                                                                                                          \\ \hline
CC0662 - Projeto de redes de computadores     & \begin{tabular}[c]{@{}l@{}}Esta diciplina é o principio da elaboração\\ de uma rede de computadores,\\ pois esta etapa é o primeiro passo\\ para na criação de uma rede,\\ onde irá servir a seu propósito\\ seja um sistema interno de uma faculdade\\  ou até de uma multinacional.\end{tabular}                                                                                          \\ \hline
CC0761 - Redes Convergentes                   & \begin{tabular}[c]{@{}l@{}}Até pouco tempo atrás, as organizações contavam com \\ redes distintas para cada meio de comunicação.\\ As redes convergentes vieram justamente para unir \\ imagens,voz e dados em uma única rede digital,\\ que atua de forma integrada.\\  Permitindo maior agilidade na comunicação e até \\ redução de custos na manutenção de estrutura de TI.\end{tabular} \\ \hline
CC0561 - Programação para dispositivos móveis & \begin{tabular}[c]{@{}l@{}}Rede de computadores é a base no quesito\\ de dispositivos móveis, \\ pois garante a integralização, conexão e a\\  comunicação destes\\  dispositivos com os servidores de internet.\end{tabular}                                                                                                                                                                \\ \hline
\end{tabular}
\end{table}

\bibliographystyle{ieeetr}
\bibliography{main.bib}

\end{document}
